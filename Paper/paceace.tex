


\documentclass[conference]{IEEEtran}
% Some Computer Society conferences also require the compsoc mode option,
% but others use the standard conference format.
%
% If IEEEtran.cls has not been installed into the LaTeX system files,
% manually specify the path to it like:
% \documentclass[conference]{../sty/IEEEtran}





% Some very useful LaTeX packages include:
% (uncomment the ones you want to load)


% *** MISC UTILITY PACKAGES ***
%
%\usepackage{ifpdf}
% Heiko Oberdiek's ifpdf.sty is very useful if you need conditional
% compilation based on whether the output is pdf or dvi.
% usage:
% \ifpdf
%   % pdf code
% \else
%   % dvi code
% \fi
% The latest version of ifpdf.sty can be obtained from:
% http://www.ctan.org/pkg/ifpdf
% Also, note that IEEEtran.cls V1.7 and later provides a builtin
% \ifCLASSINFOpdf conditional that works the same way.
% When switching from latex to pdflatex and vice-versa, the compiler may
% have to be run twice to clear warning/error messages.






% *** CITATION PACKAGES ***
%
\usepackage{cite}
% cite.sty was written by Donald Arseneau
% V1.6 and later of IEEEtran pre-defines the format of the cite.sty package
% \cite{} output to follow that of the IEEE. Loading the cite package will
% result in citation numbers being automatically sorted and properly
% "compressed/ranged". e.g., [1], [9], [2], [7], [5], [6] without using
% cite.sty will become [1], [2], [5]--[7], [9] using cite.sty. cite.sty's
% \cite will automatically add leading space, if needed. Use cite.sty's
% noadjust option (cite.sty V3.8 and later) if you want to turn this off
% such as if a citation ever needs to be enclosed in parenthesis.
% cite.sty is already installed on most LaTeX systems. Be sure and use
% version 5.0 (2009-03-20) and later if using hyperref.sty.
% The latest version can be obtained at:
% http://www.ctan.org/pkg/cite
% The documentation is contained in the cite.sty file itself.






% *** GRAPHICS RELATED PACKAGES ***
%
\ifCLASSINFOpdf
  % \usepackage[pdftex]{graphicx}
  % declare the path(s) where your graphic files are
  % \graphicspath{{../pdf/}{../jpeg/}}
  % and their extensions so you won't have to specify these with
  % every instance of \includegraphics
  % \DeclareGraphicsExtensions{.pdf,.jpeg,.png}
\else
  % or other class option (dvipsone, dvipdf, if not using dvips). graphicx
  % will default to the driver specified in the system graphics.cfg if no
  % driver is specified.
  % \usepackage[dvips]{graphicx}
  % declare the path(s) where your graphic files are
  % \graphicspath{{../eps/}}
  % and their extensions so you won't have to specify these with
  % every instance of \includegraphics
  % \DeclareGraphicsExtensions{.eps}
\fi
% graphicx was written by David Carlisle and Sebastian Rahtz. It is
% required if you want graphics, photos, etc. graphicx.sty is already
% installed on most LaTeX systems. The latest version and documentation
% can be obtained at: 
% http://www.ctan.org/pkg/graphicx
% Another good source of documentation is "Using Imported Graphics in
% LaTeX2e" by Keith Reckdahl which can be found at:
% http://www.ctan.org/pkg/epslatex
%
% latex, and pdflatex in dvi mode, support graphics in encapsulated
% postscript (.eps) format. pdflatex in pdf mode supports graphics
% in .pdf, .jpeg, .png and .mps (metapost) formats. Users should ensure
% that all non-photo figures use a vector format (.eps, .pdf, .mps) and
% not a bitmapped formats (.jpeg, .png). The IEEE frowns on bitmapped formats
% which can result in "jaggedy"/blurry rendering of lines and letters as
% well as large increases in file sizes.
%
% You can find documentation about the pdfTeX application at:
% http://www.tug.org/applications/pdftex





% *** MATH PACKAGES ***
%
\usepackage{amsmath}
% A popular package from the American Mathematical Society that provides
% many useful and powerful commands for dealing with mathematics.
%
% Note that the amsmath package sets \interdisplaylinepenalty to 10000
% thus preventing page breaks from occurring within multiline equations. Use:
%\interdisplaylinepenalty=2500
% after loading amsmath to restore such page breaks as IEEEtran.cls normally
% does. amsmath.sty is already installed on most LaTeX systems. The latest
% version and documentation can be obtained at:
% http://www.ctan.org/pkg/amsmath





% *** SPECIALIZED LIST PACKAGES ***
%
%\usepackage{algorithmic}
% algorithmic.sty was written by Peter Williams and Rogerio Brito.
% This package provides an algorithmic environment fo describing algorithms.
% You can use the algorithmic environment in-text or within a figure
% environment to provide for a floating algorithm. Do NOT use the algorithm
% floating environment provided by algorithm.sty (by the same authors) or
% algorithm2e.sty (by Christophe Fiorio) as the IEEE does not use dedicated
% algorithm float types and packages that provide these will not provide
% correct IEEE style captions. The latest version and documentation of
% algorithmic.sty can be obtained at:
% http://www.ctan.org/pkg/algorithms
% Also of interest may be the (relatively newer and more customizable)
% algorithmicx.sty package by Szasz Janos:
% http://www.ctan.org/pkg/algorithmicx




% *** ALIGNMENT PACKAGES ***
%
%\usepackage{array}
% Frank Mittelbach's and David Carlisle's array.sty patches and improves
% the standard LaTeX2e array and tabular environments to provide better
% appearance and additional user controls. As the default LaTeX2e table
% generation code is lacking to the point of almost being broken with
% respect to the quality of the end results, all users are strongly
% advised to use an enhanced (at the very least that provided by array.sty)
% set of table tools. array.sty is already installed on most systems. The
% latest version and documentation can be obtained at:
% http://www.ctan.org/pkg/array


% IEEEtran contains the IEEEeqnarray family of commands that can be used to
% generate multiline equations as well as matrices, tables, etc., of high
% quality.




% *** SUBFIGURE PACKAGES ***
%\ifCLASSOPTIONcompsoc
%  \usepackage[caption=false,font=normalsize,labelfont=sf,textfont=sf]{subfig}
%\else
%  \usepackage[caption=false,font=footnotesize]{subfig}
%\fi
% subfig.sty, written by Steven Douglas Cochran, is the modern replacement
% for subfigure.sty, the latter of which is no longer maintained and is
% incompatible with some LaTeX packages including fixltx2e. However,
% subfig.sty requires and automatically loads Axel Sommerfeldt's caption.sty
% which will override IEEEtran.cls' handling of captions and this will result
% in non-IEEE style figure/table captions. To prevent this problem, be sure
% and invoke subfig.sty's "caption=false" package option (available since
% subfig.sty version 1.3, 2005/06/28) as this is will preserve IEEEtran.cls
% handling of captions.
% Note that the Computer Society format requires a larger sans serif font
% than the serif footnote size font used in traditional IEEE formatting
% and thus the need to invoke different subfig.sty package options depending
% on whether compsoc mode has been enabled.
%
% The latest version and documentation of subfig.sty can be obtained at:
% http://www.ctan.org/pkg/subfig




% *** FLOAT PACKAGES ***
%
%\usepackage{fixltx2e}
% fixltx2e, the successor to the earlier fix2col.sty, was written by
% Frank Mittelbach and David Carlisle. This package corrects a few problems
% in the LaTeX2e kernel, the most notable of which is that in current
% LaTeX2e releases, the ordering of single and double column floats is not
% guaranteed to be preserved. Thus, an unpatched LaTeX2e can allow a
% single column figure to be placed prior to an earlier double column
% figure.
% Be aware that LaTeX2e kernels dated 2015 and later have fixltx2e.sty's
% corrections already built into the system in which case a warning will
% be issued if an attempt is made to load fixltx2e.sty as it is no longer
% needed.
% The latest version and documentation can be found at:
% http://www.ctan.org/pkg/fixltx2e


%\usepackage{stfloats}
% stfloats.sty was written by Sigitas Tolusis. This package gives LaTeX2e
% the ability to do double column floats at the bottom of the page as well
% as the top. (e.g., "\begin{figure*}[!b]" is not normally possible in
% LaTeX2e). It also provides a command:
%\fnbelowfloat
% to enable the placement of footnotes below bottom floats (the standard
% LaTeX2e kernel puts them above bottom floats). This is an invasive package
% which rewrites many portions of the LaTeX2e float routines. It may not work
% with other packages that modify the LaTeX2e float routines. The latest
% version and documentation can be obtained at:
% http://www.ctan.org/pkg/stfloats
% Do not use the stfloats baselinefloat ability as the IEEE does not allow
% \baselineskip to stretch. Authors submitting work to the IEEE should note
% that the IEEE rarely uses double column equations and that authors should try
% to avoid such use. Do not be tempted to use the cuted.sty or midfloat.sty
% packages (also by Sigitas Tolusis) as the IEEE does not format its papers in
% such ways.
% Do not attempt to use stfloats with fixltx2e as they are incompatible.
% Instead, use Morten Hogholm'a dblfloatfix which combines the features
% of both fixltx2e and stfloats:
%
% \usepackage{dblfloatfix}
% The latest version can be found at:
% http://www.ctan.org/pkg/dblfloatfix




% *** PDF, URL AND HYPERLINK PACKAGES ***
%
%\usepackage{url}
% url.sty was written by Donald Arseneau. It provides better support for
% handling and breaking URLs. url.sty is already installed on most LaTeX
% systems. The latest version and documentation can be obtained at:
% http://www.ctan.org/pkg/url
% Basically, \url{my_url_here}.





% *** Do not adjust lengths that control margins, column widths, etc. ***
% *** Do not use packages that alter fonts (such as pslatex).         ***
% There should be no need to do such things with IEEEtran.cls V1.6 and later.
% (Unless specifically asked to do so by the journal or conference you plan
% to submit to, of course. )


% correct bad hyphenation here
\hyphenation{op-tical net-works semi-conduc-tor}


\usepackage{pgfplots}
\pgfplotsset{compat=1.7}
\usepackage{pgfplotstable}
\pgfplotstableset{col sep=comma}
\newcommand{\APW}{\textit{APW}}
\newcommand{\VPW}{\textit{VPW}}
\newcommand{\APH}{\textit{APH}}
\newcommand{\VPH}{\textit{VPH}}
\DeclareMathOperator{\sgn}{sign}

\begin{document}
%
% paper title
% Titles are generally capitalized except for words such as a, an, and, as,
% at, but, by, for, in, nor, of, on, or, the, to and up, which are usually
% not capitalized unless they are the first or last word of the title.
% Linebreaks \\ can be used within to get better formatting as desired.
% Do not put math or special symbols in the title.
\title{Real-Time, Nonparametric Algorithm to Learn
Parameters for Pacemaker Beat Detection}


% author names and affiliations
% use a multiple column layout for up to three different
% affiliations
\author{\IEEEauthorblockN{Michael Shell}
\IEEEauthorblockA{School of Electrical and\\Computer Engineering\\
	Georgia\cite{Nobody06} Institute of Technology\\
Atlanta, Georgia 30332--0250\\
Email: http://www.michaelshell.org/contact.html}
\and
\IEEEauthorblockN{Homer Simpson}
\IEEEauthorblockA{Twentieth Century Fox\\
Springfield, USA\\
Email: homer@thesimpsons.com}
\and
\IEEEauthorblockN{James Kirk\\ and Montgomery Scott}
\IEEEauthorblockA{Starfleet Academy\\
San Francisco, California 96678--2391\\
Telephone: (800) 555--1212\\
Fax: (888) 555--1212}}

% conference papers do not typically use \thanks and this command
% is locked out in conference mode. If really needed, such as for
% the acknowledgment of grants, issue a \IEEEoverridecommandlockouts
% after \documentclass

% for over three affiliations, or if they all won't fit within the width
% of the page, use this alternative format:
% 
%\author{\IEEEauthorblockN{Michael Shell\IEEEauthorrefmark{1},
%Homer Simpson\IEEEauthorrefmark{2},
%James Kirk\IEEEauthorrefmark{3}, 
%Montgomery Scott\IEEEauthorrefmark{3} and
%Eldon Tyrell\IEEEauthorrefmark{4}}
%\IEEEauthorblockA{\IEEEauthorrefmark{1}School of Electrical and Computer Engineering\\
%Georgia Institute of Technology,
%Atlanta, Georgia 30332--0250\\ Email: see http://www.michaelshell.org/contact.html}
%\IEEEauthorblockA{\IEEEauthorrefmark{2}Twentieth Century Fox, Springfield, USA\\
%Email: homer@thesimpsons.com}
%\IEEEauthorblockA{\IEEEauthorrefmark{3}Starfleet Academy, San Francisco, California 96678-2391\\
%Telephone: (800) 555--1212, Fax: (888) 555--1212}
%\IEEEauthorblockA{\IEEEauthorrefmark{4}Tyrell Inc., 123 Replicant Street, Los Angeles, California 90210--4321}}




% use for special paper notices
%\IEEEspecialpapernotice{(Invited Paper)}




% make the title area
\maketitle

% As a general rule, do not put math, special symbols or citations
% in the abstract
\begin{abstract}
While beat detection in an Electrocardiogram
(ECG) signal is a well-studied problem, we propose a novel
algorithm, within the context of pacemakers, that learns the
heights and widths of atrial and ventricular peaks from simply
processing 10 seconds of ECG data sampled at 1 kHz. Utilizing
a purely data-driven solution to learn the parameters of atrial
and ventricular peaks will allow pacemakers to set their own
detection parameters for a specific patient and adaptively tune
their detection parameters, for that specific patient, over time.
We have validated these results on 51 separate channels of ECG
data. Additionally, we have implemented the algorithm on a
Field Programmable Gate Array and tested it on a Langdendorff
heart to illustrate that our algorithm can be implemented on
hardware and run in real time.
\end{abstract}

% no keywords




% For peer review papers, you can put extra information on the cover
% page as needed:
% \ifCLASSOPTIONpeerreview
% \begin{center} \bfseries EDICS Category: 3-BBND \end{center}
% \fi
%
% For peerreview papers, this IEEEtran command inserts a page break and
% creates the second title. It will be ignored for other modes.
\IEEEpeerreviewmaketitle



\section{Introduction}
% no \IEEEPARstart
Cardiac diseases are the number one cause of death in
the United States. More than 600,000 people each year
die due to some failure in the heart~\cite{death-stats}. A healthy heart
functions through a regular cardiac cycle, where the pace
making neurons in the sinoatrial (SA) node of the heart
generate an electrical signal, which travels from the atria
down to the ventricles, to stimulate heart contraction.
Many heart failures result from the heart's inability to
generate or conduct these electrical signals and stimulate
muscle contraction properly. To combat this, a
combination of researchers and doctors developed a
device called the pacemaker that can stimulate the heart
to contract artificially through externally supplied
electrical pulses.
Current pacemakers utilize the state of the art
algorithm called DDDR in order to determine whether
the heart requires stimulation~\cite{basic-pacing}. The devices utilize
ECG beat detection algorithms to determine whether a
heartbeat has occurred. If it does not see a heartbeat
within a certain period of time, it will deliver an
electrical stimulation to the heart.
ECG beat detection itself is a well-studied problem, as
several algorithms have been proposed to identify the
beats within an ECG signal~\cite{realtime-qrs, ecg-filter}. However, current
algorithms perform beat detection without distinguishing
which part of the ECG signal corresponds to the atrial
portion of the heartbeat and which part corresponds to
the ventricular portion of the heartbeat. As a result,
current pacemakers often times struggle to distinguish
between atrial and ventricular beats: information that is
crucial for Cardiac Resynchronization Therapy (CRT)~\cite{multisite-crt}.
In fact, up to 30\% of patients with pacemakers do
not respond to CRT implemented by current pacemakers~\cite{multisite-crt}.
To make ECG beat detection more suitable for CRT,
we propose a nonparametric algorithm that learns the
height of an atrial peak (\APH), the height of a ventricular
peak (\VPH), the width of an atrial peak (\APW), and the
width of a ventricular peak (\VPW) within an ECG signal
by simply processing 10 seconds of ECG data sampled
at 1 kHz. With these parameters, we then implement
atrial and ventricular beat detection to illustrate that our
learned parameters can be used to distinguish between
atrial and ventricular beats within an ECG signal.
Additionally, we implement the algorithm on a Field
Programmable Gate Array to demonstrate that the
algorithm can be run on a real-time system.
Sections II and III thoroughly explain our approach
for learning \APH{}, \VPH{}, \APW{}, and \VPW{}. Section IV
details the hardware implementation of the algorithm.
Finally, Section V documents our testing procedure and
validates our algorithmic results and hardware system.

\section{Atrial and Ventricular Peak Width Learning}
Our algorithm begins by learning \APW{} and \VPW{}. We
detail our approach in the following sections.

\subsection{Finding Peak Shaped Patterns in the ECG Signal}
Let $f[n]$ be a 10 second ECG signal sampled at 1 kHz.
We begin by computing the weighted time
derivative of $f[n]$, denoted $w[n]$.
\begin{equation}
	w[n]=(f[n]-f[n-1]) * \left|f[n] - f[n-1]\right| * \left|f[n]\right|
\end{equation}

% TODO: Give some intuition for why this is good.
Next, the algorithm searches for when the local
maxima and minima of $w[n]$ occur, within a window of
200 ms. Thus, we compute:
\begin{equation}
	r[n] = \left \{
		\begin{array}{lr}
			1 & \text{if } w[n] = \max\limits_{-100 \le i \le 100} (w[n+i]) \\
			-1 & \text{if } w[n] = \min\limits_{-100 \le i \le 100} (w[n+i]) \\
			0 & \text{otherwise}
		\end{array}
	\right.
\end{equation}
Note, that a value of 1 in $r[n]$ corresponds to the
steepest rising edge of peaks in $f[n]$, while a value of -1
in $r[n]$ corresponds to the steepest falling edge of a peak
in $f[n]$. Thus, $r[n]$ identifies all peak like structures in
$f[n]$.  \figurename~\ref{fig:fwr} illustrates this concept.
\begin{figure}
	\centering
	\begin{tikzpicture}
		\begin{axis}[
				%title      =,
				xlabel     = Time (ms),
				ylabel     = Voltage (mV),
				legend pos = north east,
			]
			\addplot+[mark=none, color=red, smooth] table[x = t, y = f, col sep=comma]{example1.csv} ;
			\addplot+[mark=none, color=blue, smooth] table[x = t, y = w, col sep=comma]{example1.csv} ;
			\addplot+[only marks, mark=o, mark size = 2pt,color=black] table[x = t, y = r, col sep=comma]{example1.csv} ;
			\legend{$f[n]$, $w[n]$, $r[n]$}
		\end{axis}
	\end{tikzpicture}
	\caption{
	Illustration of the relationship between $f[n]$, $w[n]$ and $r[n]$}
	\label{fig:fwr}
\end{figure}

\subsection{Featurizing Peaks and Clustering to Determine \APW{} and \VPW}
Using $r[n]$, we identify a peak, $P_i$, to be a rising edge
followed in less than 75ms by a falling edge. Let the
rising edge and falling edges of $P_i$ occur at time $m$ and $n$
respectively. The peak is then featurized by its width and
height as follows:
\begin{equation}
	P_i = \left\{n-m, \left( \max\limits_{m\le j\le n} f[j] \right) - \frac{f[n]+f[m]}{2} \right\}
\end{equation}
Thus a two-dimensional data point characterizes each
peak. We then construct $\mathbf{P} = [P_1, P_2, \dots P_n]$ and then
cluster the peaks in $\mathbf{P}$ using a standard two cluster k-means
algorithm~\cite{k-means}. The ventricular cluster will have a
center with a greater height to width ratio, since
ventricles are generally taller and thinner. Let $C_v$ and $C_A$
be the centers of the ventricular and atrial clusters
respectively. The algorithm then stores the ventricular
peak width (\VPW) and atrial peak width (\APW) as the
rounded width term of the centers as follows:
\begin{equation}
	VPW=C_V\{1\} \text{ and } APW=V_A\{1\}
\end{equation}

A visual illustration of how k-means clusters the
featurized peaks $\mathbf{P}$ into atrial and ventricular peaks and
determines the resulting centroids $C_V$ and $C_A$ can be
found in  \figurename~\ref{fig:kmeans}.


\begin{figure}
	\centering
	\begin{tikzpicture}
		\begin{axis}[
				%title      =,
				xlabel     = Peak Length (ms),
				ylabel     = Peak Height (mV),
				legend pos = north east,
			]
			\addplot+[only marks, mark=*] table[x = x, y = y, col sep=comma]{kmeans-ventricles.csv} ;
			\addplot+[only marks, mark=*] table[x = x, y = y, col sep=comma]{kmeans-atria.csv} ;
			\legend{Ventricles, Atria}
		\end{axis}
	\end{tikzpicture}
	\caption{
	K-Means clustering featurized peaks into ventricles and atria.}
		\label{fig:kmeans}
\end{figure}

Note, that the height that we use to featurize a peak in
this portion of the algorithm does not correspond to the
actual values of \APH{} and \VPH{} in the original ECG
signal $f[n]$. The `height' value used for peak
featurization simply characterizes the height of the peak
with respect to the portions of the peak during which its
value is most rapidly increasing or decreasing. It does
not give any useful information about the height of an
atrial or ventricular peak with respect to the baseline
value of 0.
\subsection{Subsection Heading Here}
Subsection text here.


\subsubsection{Subsubsection Heading Here}
Subsubsection text here.


% An example of a floating figure using the graphicx package.
% Note that \label must occur AFTER (or within) \caption.
% For figures, \caption should occur after the \includegraphics.
% Note that IEEEtran v1.7 and later has special internal code that
% is designed to preserve the operation of \label within \caption
% even when the captionsoff option is in effect. However, because
% of issues like this, it may be the safest practice to put all your
% \label just after \caption rather than within \caption{}.
%
% Reminder: the "draftcls" or "draftclsnofoot", not "draft", class
% option should be used if it is desired that the figures are to be
% displayed while in draft mode.
%
%\begin{figure}[!t]
%\centering
%\includegraphics[width=2.5in]{myfigure}
% where an .eps filename suffix will be assumed under latex, 
% and a .pdf suffix will be assumed for pdflatex; or what has been declared
% via \DeclareGraphicsExtensions.
%\caption{Simulation results for the network.}
%\label{fig_sim}
%\end{figure}

% Note that the IEEE typically puts floats only at the top, even when this
% results in a large percentage of a column being occupied by floats.


% An example of a double column floating figure using two subfigures.
% (The subfig.sty package must be loaded for this to work.)
% The subfigure \label commands are set within each subfloat command,
% and the \label for the overall figure must come after \caption.
% \hfil is used as a separator to get equal spacing.
% Watch out that the combined width of all the subfigures on a 
% line do not exceed the text width or a line break will occur.
%
%\begin{figure*}[!t]
%\centering
%\subfloat[Case I]{\includegraphics[width=2.5in]{box}%
%\label{fig_first_case}}
%\hfil
%\subfloat[Case II]{\includegraphics[width=2.5in]{box}%
%\label{fig_second_case}}
%\caption{Simulation results for the network.}
%\label{fig_sim}
%\end{figure*}
%
% Note that often IEEE papers with subfigures do not employ subfigure
% captions (using the optional argument to \subfloat[]), but instead will
% reference/describe all of them (a), (b), etc., within the main caption.
% Be aware that for subfig.sty to generate the (a), (b), etc., subfigure
% labels, the optional argument to \subfloat must be present. If a
% subcaption is not desired, just leave its contents blank,
% e.g., \subfloat[].


% An example of a floating table. Note that, for IEEE style tables, the
% \caption command should come BEFORE the table and, given that table
% captions serve much like titles, are usually capitalized except for words
% such as a, an, and, as, at, but, by, for, in, nor, of, on, or, the, to
% and up, which are usually not capitalized unless they are the first or
% last word of the caption. Table text will default to \footnotesize as
% the IEEE normally uses this smaller font for tables.
% The \label must come after \caption as always.
%
%\begin{table}[!t]
%% increase table row spacing, adjust to taste
%\renewcommand{\arraystretch}{1.3}
% if using array.sty, it might be a good idea to tweak the value of
% \extrarowheight as needed to properly center the text within the cells
%\caption{An Example of a Table}
%\label{table_example}
%\centering
%% Some packages, such as MDW tools, offer better commands for making tables
%% than the plain LaTeX2e tabular which is used here.
%\begin{tabular}{|c||c|}
%\hline
%One & Two\\
%\hline
%Three & Four\\
%\hline
%\end{tabular}
%\end{table}


% Note that the IEEE does not put floats in the very first column
% - or typically anywhere on the first page for that matter. Also,
% in-text middle ("here") positioning is typically not used, but it
% is allowed and encouraged for Computer Society conferences (but
% not Computer Society journals). Most IEEE journals/conferences use
% top floats exclusively. 
% Note that, LaTeX2e, unlike IEEE journals/conferences, places
% footnotes above bottom floats. This can be corrected via the
% \fnbelowfloat command of the stfloats package.




\section{Conclusion}
The conclusion goes here.




% conference papers do not normally have an appendix


% use section* for acknowledgment
\section*{Acknowledgment}


The authors would like to thank...





% trigger a \newpage just before the given reference
% number - used to balance the columns on the last page
% adjust value as needed - may need to be readjusted if
% the document is modified later
%\IEEEtriggeratref{8}
% The "triggered" command can be changed if desired:
%\IEEEtriggercmd{\enlargethispage{-5in}}

% references section

% can use a bibliography generated by BibTeX as a .bbl file
% BibTeX documentation can be easily obtained at:
% http://mirror.ctan.org/biblio/bibtex/contrib/doc/
% The IEEEtran BibTeX style support page is at:
% http://www.michaelshell.org/tex/ieeetran/bibtex/
\bibliographystyle{IEEEtran}
% argument is your BibTeX string definitions and bibliography database(s)
\bibliography{IEEEabrv,sources}
%
% <OR> manually copy in the resultant .bbl file
% set second argument of \begin to the number of references
% (used to reserve space for the reference number labels box)
%\begin{thebibliography}{1}
%
%\bibitem{IEEEhowto:kopka}
%H.~Kopka and P.~W. Daly, \emph{A Guide to \LaTeX}, 3rd~ed.\hskip 1em plus
%  0.5em minus 0.4em\relax Harlow, England: Addison-Wesley, 1999.

% \end{thebibliography}




% that's all folks
\end{document}


